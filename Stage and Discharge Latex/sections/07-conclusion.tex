\documentclass[../main.tex]{subfiles}
\begin{document}
\section{Conclusion}
Durante el trabajo se consideraron varias alternativas, dando la oportunidad de estudiar los datos a un fondo sorprendente, e impulsando a experimentar con soluciones creativas e innovadoras. La solución que mejores resultados fue tomar los datos y dividirlos por temporadas (primavera, verano, otoño e invierno), especializando un modelo para cada temporada, a su vez se implementó una solución que solo necesitaría imágenes con un leve procesamiento sin necesidad de un análisis complejo de las imágenes. Aunque el anterior muestra inconsistencia de resultados precisos se considera que fue la solución más creativa debido a que una vez entrenado podrá funcionar sin necesidad de alteraciones al sitio, implementando la solución de la hipótesis que creamos desde un inicio. Para concluir confirmamos la hipótesis inicial en la cual se creía que un modelo entrenado para temporadas específicas demostraría mejores resultados. Con esto aclarado los resultados con mayor precisión y consistencia de buenos resultados fue el modelo de RFR aplicado desde un comienzo por el equipo de Nebraska con la alteración de datos y segmentación de conjunto de entrenamiento y prueba.


\end{document}

\clearpage